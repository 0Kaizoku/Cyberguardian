\documentclass[conference]{IEEEtran}
\usepackage[utf8]{inputenc}
\usepackage[T1]{fontenc}
\usepackage[french]{babel}
\usepackage{graphicx}
\usepackage{amsmath}
\usepackage{booktabs}
\usepackage{hyperref}
\usepackage{cite}

\title{CyberGuardian:\\Protection des Données Personnelles sur Mobile}

\author{
  \IEEEauthorblockN{Amine \textit{et al.}}
  \IEEEauthorblockA{
    Département Informatique, Université X\\
    \{amine, coauthor\}@exemple.edu
  }
}

\begin{document}
\maketitle

\begin{abstract}
Nous présentons CyberGuardian, une application Android et un service backend
de détection des applications malveillantes et abusives. 
Notre approche combine analyse statique des permissions et détection comportementale à 
l’aide d’un modèle XGBoost, exposé via une microservice FastAPI, et un serveur Spring Boot. 
Les résultats expérimentaux montrent une très haute précision (99\,\%) sur un jeu de test.
\end{abstract}

\section{Introduction}
La prolifération des smartphones entraîne une explosion du nombre d'applications malveillantes
et abusives. CyberGuardian vise à protéger la vie privée des utilisateurs
en détectant à la fois les permissions excessives (analyse statique) 
et les comportements anormaux à l'exécution (analyse dynamique).

\section{Architecture Système}
\begin{figure}[h]
  \centering
  \includegraphics[width=0.9\linewidth]{figures/architecture_diagram.png}
  \caption{Architecture générale de CyberGuardian.}\label{fig:arch}
\end{figure}

Le système se compose de :
\begin{itemize}
  \item \textbf{Client Android}  
    \begin{itemize}
      \item Scanner des applications et de leurs permissions.
      \item Module statique de détection (règles de risque sur permissions).
      \item Module dynamique d’observation via \texttt{AccessibilityService} et \texttt{UsageStatsManager}.
      \item Envoi des données au backend via Retrofit.
    \end{itemize}
  \item \textbf{Backend Spring Boot}  
    \begin{itemize}
      \item API REST \texttt{/api/analyzeApp} accueillant les données du client.
      \item Appel à la microservice FastAPI pour la prédiction XGBoost.
      \item Stockage des résultats dans PostgreSQL.
    \end{itemize}
  \item \textbf{Microservice FastAPI}  
    \begin{itemize}
      \item Chargement du modèle XGBoost pré-entraîné.
      \item Endpoint \texttt{/predict} renvoyant \texttt{risk\_score} et \texttt{risk\_label}.
      \item Lien prévu pour enrichissement futur avec API VirusTotal.
    \end{itemize}
\end{itemize}

\section{Mise en œuvre}

\subsection{Analyse Statique}
Nous définissons un score statique basé sur la présence de permissions dangereuses.  
Exemple de règle :  
\[
  \text{Score}_{\mathrm{static}} = \sum_{p \in P_{\mathrm{app}}} w_p
\]
où chaque permission $p$ reçoit un poids $w_p$ (ex. INTERNET : 1, READ\_SMS : 2, etc.).

\subsection{Analyse Dynamique}
L’application Android surveille :
\begin{itemize}
  \item Les événements d’interface (\texttt{onAccessibilityEvent}).
  \item Les statisques d’usage (\texttt{UsageStatsManager}).
\end{itemize}
Ces événements forment un vecteur de caractéristiques temporelles, chargé ensuite vers le backend.

\subsection{Modèle XGBoost}
Le modèle est entraîné hors ligne avec un jeu de données d’environ \num{10000} applications,  
balayant $n$ permissions et $m$ métriques dynamiques.  
Sa performance est évaluée via :
\begin{itemize}
  \item Matrice de confusion.
  \item Courbe ROC.
  \item AUC = 0.99.
\end{itemize}

\section{Résultats Expérimentaux}

\subsection{Distribution des permissions dangereuses}
\begin{figure}[h]
  \centering
  \includegraphics[width=\linewidth]{figures/permission_histogram.png}
  \caption{Histogramme du nombre de permissions considérées ``dangereuses'' par application.}\label{fig:hist}
\end{figure}

\subsection{Performance du modèle}\label{sec:perf}
\begin{figure}[h]
  \centering
  \includegraphics[width=\linewidth]{figures/roc_curve.png}
  \caption{Courbe ROC du modèle XGBoost. AUC = 0.99.}\label{fig:roc}
\end{figure}

\section{Conclusion}
Nous avons présenté un système hybride combinant règles statiques et apprentissage
automatique pour la détection de logiciels malveillants sur Android.
CyberGuardian atteint une très haute fiabilité tout en restant extensible
(p. ex. intégration future de VirusTotal).  

\section*{Remerciements}
Ce travail s’appuie sur les recherches de Burguera et al. (Crowdroid) \cite{Burguera2010Crowdroid}.

\bibliographystyle{IEEEtran}
\begin{thebibliography}{1}
\bibitem{Burguera2010Crowdroid}
I.~Burguera, U.~Zurutuza, and S.~Nadjm-Tehrani, ``Crowdroid: Behavior-Based Malware Detection System for Android,'' in \emph{Proc. of the 1st ACM Workshop on Security and Privacy in Smartphones}, 2010.
% Ajoutez ici vos autres références
\end{thebibliography}

\end{document}
